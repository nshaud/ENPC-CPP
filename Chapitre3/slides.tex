\documentclass{beamer}
\usepackage[french]{babel}
\usepackage{hyperref}
\usepackage{graphicx}
\usepackage{amsmath,amssymb}
\usepackage{tabularx}
\usepackage{booktabs}
\usepackage[compatibility=false]{caption}
\usepackage[toc,page]{appendix}
\usepackage{minted}

\makeatletter
  \def\beamer@calltheme#1#2#3{%
    \def\beamer@themelist{#2}
    \@for\beamer@themename:=\beamer@themelist\do
    {\usepackage[{#1}]{\beamer@themelocation/#3\beamer@themename}}}

  \def\usefolder#1{
    \def\beamer@themelocation{#1}
  }
  \def\beamer@themelocation{}


\newcolumntype{Y}{>{\centering\arraybackslash}X}

\usefolder{../theme}
\usetheme[numbering=fraction,block=fill,progressbar=frametitle]{metropolis} %Use metropolis theme

\definecolor{bg}{rgb}{0.95,0.95,0.95}
\setminted{bgcolor=bg,fontsize=\scriptsize,autogobble,mathescape,breaklines,tabsize=2}
\setmintedinline{breaklines,autogobble,fontsize=\scriptsize}

\begin{document}

\title[C++]{Introduction à la programmation en C++}
\author[nicolas.audebert@onera.fr]{Nicolas Audebert}
\setmainfont{Fira Sans}


\AtBeginSection[]{
  \begin{frame}{Plan de la séance}
  \small \tableofcontents[currentsection]
  \end{frame}
}

\subtitle{Variables, tests, boucles et fonctions}
\date{}
\maketitle

\section{Variables}

\begin{frame}[fragile]
	\frametitle{Type des variables}
	\begin{block}{Langage typé}
        Dans un \textbf{langage typé}, les variables appartiennent à un espace donné et ne peuvent en changer. C++ est un langage typé.
	\end{block}
	\vfill
	\begin{minipage}{0.38\linewidth}
		\begin{minted}{cpp}
1: int i;
2: i = 2;
3: float d = 5.6;
4: char a = 'z';
5: bool v = true;
		\end{minted}
	\end{minipage}
	\hfill
	\begin{minipage}{0.58\linewidth}
		\begin{enumerate}
			\item Déclaration d'un entier \texttt{\textbf{i}}
            \item Affectation de la valeur 2 à \texttt{\textbf{i}}
			\item Déclaration et affectation d'un réel \texttt{\textbf{d}}
            \item Idem pour un caractère \texttt{\textbf{a}}
            \item Idem pour un booléen \texttt{\textbf{v}}
		\end{enumerate}
	\end{minipage}
\end{frame}

\begin{frame}[fragile]
	\frametitle{Déclaration et affectation d'une variable}
	\begin{block}{Déclaration}
		\centering
		\texttt{\textbf{type} nom\_de\_variable \textbf{\huge ;}}
	\end{block}
	\begin{block}{Affectation}
		\centering
		\texttt{nom\_de\_variable = valeur \textbf{\huge ;}}
	\end{block}
	\vspace{0.5cm}
	Il est possible de faire les deux en même temps :
	\begin{center}
        \mintinline{cpp}{type nom_de_variable = valeur ;}
	\end{center}

	\begin{minipage}{0.40\linewidth}
		\begin{minted}{python}
a = 2 # un entier
a = 3.4 # un reel
		\end{minted}
	\end{minipage}
	\hfill
	\begin{minipage}{0.57\linewidth}
		\begin{minted}{cpp}
a = 2; // ERREUR: a n'existe pas
int a; // Déclaration 
a = 2; // Affectation
double pi = 3.14;
		\end{minted}
	\end{minipage}
\end{frame}

\begin{frame}
	\frametitle{Les types natifs}
	\begin{tabular}{|r|c|c|c|}
		\hline
		\textbf{Type} & \textbf{Espace} & \textbf{min} & \textbf{max}\\
		\hline
		int & $\mathbb{Z}$ & $-2147483648$ & $2147483647$ \\
		\hline
		unsigned int & $\mathbb{N}$ & $0$ & $4294967295$ \\
		\hline
		char & $\mathbb{Z}$ & -128 & 127\\
		\hline
		unsigned char & $\mathbb{Z}$ & 0 & 255\\
		\hline
		float & $\mathbb{R}$& \multicolumn{2}{c|}{3.4E +/- 38}\\
		\hline
		double & $\mathbb{R}$& \multicolumn{2}{c|}{1.7E +/- 308} \\
		\hline
		bool & $\{0,1\}$& false (0)& true (1)\\
		\hline
	\end{tabular}
\end{frame}

\begin{frame}[fragile]
	\frametitle{Exemple}
	\begin{minted}{cpp}
int i = 2; // Définition et affectation
cout << i << endl; // Affiche "2"

int j;
j = i; // j est une nouvelle variable valant 2
i = 1; // Ne modifie que i, pas j !
cout << i << ", " << j << endl; // Affiche "1, 2"

int k, l, m = 4; // Définition multiple
k = l = 3;   // Affectation multiple
cout << k << ", " << l << ", " << m << endl; // Affiche "3, 3, 4"

int n = 5, o = n, p = INT_MAX ; // Initialisations
cout << n << ", " << o << ", " << p << endl;

int q = r = 4 ; // Erreur ! (r n'existe pas)

const int s = 12; // Définit s comme constante valant 12
s = 13; // Erreur !
	\end{minted}
\end{frame}

\begin{frame}[fragile]
	\frametitle{Conversions et règles de calcul}
	\begin{minipage}{0.42\linewidth}
	\begin{block}{Opérateurs mathématiques}
		\begin{itemize}
			\item \texttt{\textbf{+}} : addition
			\item \texttt{\textbf{-}} : soustraction
			\item \texttt{\textbf{*}} : multiplication
			\item \texttt{\textbf{/}} : division (entière ou non)
			\item \texttt{\textbf{\%}} : modulo
		\end{itemize}
	\end{block}
	\end{minipage}
	\hfill
	\begin{minipage}{0.56\linewidth}
	\begin{minted}{cpp}
int a = 15, b = 7;
int c = a/b; // c=2
int d = a % b; // d=1

double e = b/a; // ATTENTION e = 0
double f = double(b)/a; // f = 0.467
double g = (b+0.)/a // g = 0.467

float h = 5.6;
int i = h; // i = 5 ==> WARNING
int i = int(h); // i = 5, pas de warning
	\end{minted}
\end{minipage}
\end{frame}

\section{Tests}

\begin{frame}
	\frametitle{Expressions booléennes}
	Une expression booléenne est une affirmation, elle est soit \textbf{VRAIE} soit \textbf{FAUSSE}.
	\vfill
	\begin{block}{En C++}
		Le résultat d'une expression booléenne est un booléen, type \texttt{\textbf{bool}} qui prend les valeurs \texttt{\textbf{true}} ou \texttt{\textbf{false}}.
	\end{block}

\end{frame}

\begin{frame}[fragile]
	\frametitle{Opérateurs booléen et de comparaison}
	\begin{minipage}{0.43\linewidth}
		\begin{itemize}
			\item \texttt{\textbf{==}} : égalité
			\item \texttt{\textbf{!=}} : différence
			\item \texttt{\textbf{<}} (\texttt{\textbf{<=}}) : infériorité
			\item \texttt{\textbf{>}} (\texttt{\textbf{>=}}) : supériorité
			\item \texttt{\textbf{\&\&}} : ET logique
			\item \texttt{\textbf{||}} : OU logique
			\item \texttt{\textbf{!}} : NON logique
		\end{itemize}
	\end{minipage}
	\hfill
	\begin{minipage}{0.53\linewidth}
		\begin{minted}{cpp}
int a = 3, b = 5, c == 8;
bool b1 = a == b; // b1 = 0 (false)
bool b2 = (a+c) != b // b2 = 1 (true)
cout << ( c >= a ) << endl; // 1
bool b3 = b2 && (a+b == c) // b3 = 1
cout << b3 || b1 << endl; // 1
cout << ! b3 << endl; // 0
		\end{minted}
	\end{minipage}
	\begin{block}{Priorités}
		Le \textbf{ET} est prioritaire sur le \textbf{OU}
	\end{block}

	\begin{block}{a XOR b}
		\centering
		\texttt{ a \&\& (!b) || (!a) \&\& b }
	\end{block}

\end{frame}

\begin{frame}[fragile]
	\frametitle{SI (\dots) FAIRE \dots SINON \dots}
	\begin{block}{Syntaxe}
		\centering
		\texttt{\textbf{if}(expression booléenne)\{...\} \textbf{else} \{...\}}
	\end{block}
	\begin{minted}{cpp}
int annee;
cout << "Entrez une année" << endl;
cin >> annee;

if( annee % 4 == 0){
    if(annee % 100 == 0 && annee%400 != 0){
        cout << "Cette année n'est pas bissextile" << endl;
    } else {
        cout << "Année bissextile" << endl;
    }
} else {
    cout << "Cette année n'est pas bissextile" << endl;
}
	\end{minted}
\end{frame}

\begin{frame}[fragile]
	\frametitle{Le switch}
	Le \texttt{\textbf{switch}} permet une énumération de ce qu'il faut faire en fonction des valeurs d'une variable \textbf{entière}.
    \vspace{-0.3em}
	\begin{minipage}{0.47\linewidth}
		\begin{minted}{cpp}
char c;
if (c == 'a'){
    cout << "Lettre 'a'";
}
else if (c == 'f'){
    cout << "Lettre 'f'";
} else if(c=='e'||c=='i'||c=='o'
             ||c=='u'||c=='y') {
    cout << "Autre voyelle";
} else {
    cout << "Autre chose";
}
		\end{minted}
	\end{minipage}
	\hfill
	\begin{minipage}{0.47\linewidth}
		\begin{minted}{cpp}
char c;
switch ( c ) {
case 'a' :
    cout << "Lettre 'a'";
    break ;
case 'f' :
    cout << "Lettre 'f'";
    break ;
case 'e' :
case 'i' :
case 'o' :
case 'u' :
case 'y' :
    cout<<"Autre voyelle" ;
    break ;
default :
    cout<<"Une consonne";
    break ;
		\end{minted}
	\end{minipage}
\end{frame}

\section{Portée}

\begin{frame}[fragile]
	\frametitle{Portée des variables}
	\begin{block}{La règles des accolades}
		Une variable n'existe que dans le bloc (et les sous-blocs), défini par des accolades, dans lequel elle a été déclarée.
	\end{block}

	\begin{minipage}{0.45\linewidth}
		\begin{minted}{python}
a = 3
b = 6
if a < b:
    print(a)
    c = a + b
    print(c) # OK
print(c) # OK, c existe en dehors de son bloc
		\end{minted}
	\end{minipage}
	\hfill
	\begin{minipage}{0.50\linewidth}
		\begin{minted}{cpp}
int a = 3, b = 6;
if(a < b){
    cout << a << endl; // OK
    int c = a + b;
    cout << c << endl; // OK
}
cout << c << endl; // ERREUR !
		\end{minted}
	\end{minipage}
\end{frame}

\section{Boucles}

\begin{frame}[fragile]
	\frametitle{TANT QUE (\dots) FAIRE \dots}

    \begin{block}{Syntaxe}
        \texttt{\textbf{while}(expression booléenne)\{...\}}\\
        \texttt{\textbf{do}\{...\}\textbf{while}(expression booléenne);}
    \end{block}

    \begin{minipage}{0.47\linewidth}
        \begin{minted}{python}
a = 30
while a>0:
    print(a)
    a -= 2

        \end{minted}
    \end{minipage}
    \hfill
    \begin{minipage}{0.47\linewidth}
        \begin{minted}{cpp}
int a = 30;
while(a > 0){
    cout << a << endl;
    a -=2;
}

int b = 1;
do {
    b *= 2;
}
while(b != 1024);

        \end{minted}
    \end{minipage}
\end{frame}

\begin{frame}[fragile]
	\frametitle{POUR (\dots) FAIRE \dots}
    \begin{block}{Syntaxe}
        \texttt{\textbf{for}(initialisation{\huge;}expression booléenne{\huge;}itération) \{...\}}\\
    \end{block}

    \begin{minipage}{0.37\linewidth}
        \begin{minted}{python}
for a in xrange(10):
    print(a)

# C'est équivalent à :

a = 0
while a < 10:
    print(a)
    a +=1
        \end{minted}
    \end{minipage}
    \hfill
    \begin{minipage}{0.57\linewidth}
        \begin{minted}{cpp}
for(int a = 0; a < 10; a++){ // a++ est a = a + 1
    cout << a << endl;
}

# C'est équivalent à :

int a = 0;
while(a < 10){
    cout << a << endl;
    a++;
}
        \end{minted}
    \end{minipage}
\end{frame}

\begin{frame}[fragile]
    \frametitle{Sortir des boucles}
    L'instruction \texttt{\textbf{break}} permet de forcer la sortie d'une boucle.
    \vfill
    \begin{minipage}{0.37\linewidth}
    \begin{minted}{cpp}
/*
Sortir d'une boucle infinie
*/
int a = 1;
while(true){ // boucle infini
    a *= 2;
    cout << a << endl;
    if(a > 5000){
        break;
    }
}
    \end{minted}
    \end{minipage}
    \hfill
    \begin{minipage}{0.57\linewidth}
    \begin{minted}{cpp}
/*
Sortir avant le terme d'un for
*/
for(int i = 0; i < 10; i++){
    int res;
    cout << "Entrez un entier" << endl;
    cin >> res;
    if(res == 5){
        cout << "5" << endl;
        break;
    }
}
    \end{minted}
    \end{minipage}

\end{frame}

\section{Fonctions}

\begin{frame}
	\frametitle{Sens mathématique}

    En C++, le concept de fonction est le même qu'en mathématiques :
    \begin{center}
        \only<1>{\includegraphics[width=0.5\linewidth]{images/fonctions_00.pdf}}
        \only<2>{\includegraphics[width=0.5\linewidth]{images/fonctions_01.pdf}}
        \only<3>{\includegraphics[width=0.5\linewidth]{images/fonctions_02.pdf}}
        \only<4>{\includegraphics[width=0.5\linewidth]{images/fonctions_03.pdf}}
    \end{center}

\end{frame}

\begin{frame}[fragile]
\frametitle{Exemple}

\begin{block}{La fonction signe}
$$
\begin{array}{cccc}
signe: & \mathbb{R} & \to & \{-1,0,1\} \\
& x & \mapsto & \left\{
    \begin{split}
    -1 \: \text{si} \: x < 0\\
    1 \: \text{si}  \: x > 0\\
    0 \: \text{sinon}
    \end{split}
  \right.
\end{array}
$$
\end{block}
Deux manières d'écrire la même fonction.
\begin{minipage}{0.47\linewidth}
\begin{minted}{cpp}
int signe(double x){
  int s = 0;
	  
  if(x < 0)
    s = -1;
  if(x > 0)
    s = 1;
  return s;
}
\end{minted}
\end{minipage}
\hfill
\begin{minipage}{0.47\linewidth}
\begin{minted}{cpp}
int signe(double x){
    if(x < 0)
        return -1;
    if(x > 0)
        return 1;
    return 0;
}
\end{minted}
\end{minipage}
\end{frame}

\begin{frame}[fragile]
	\frametitle{Fonction sans retour}
	Pour des fonctions qui ne renvoient rien, on utilise le type de retour vide : \texttt{\textbf{void}}.

	Le retour vide : \texttt{\textbf{return;}} est optionnel.

	\begin{block}{Afficher}
	Afficher les coordonnées $(x,y,z)$ d'un vecteur :
		$$
		\begin{array}{cccc}
		affiche: & \mathbb{R}^3 & \to & \emptyset \\
		\end{array}
		$$
	\end{block}

	\begin{minted}{cpp}
void affiche(double x, double y, double z){
    cout << "(" << x << "," << y << "," << z << ")" << endl;
    return; // OPTIONNEL
}
void affiche(double x, double y, double z){
    cout << "(" << x << "," << y << "," << z << ")" << endl;
}
	\end{minted}
\end{frame}

\begin{frame}[fragile]
	\frametitle{Limitations}
	\begin{itemize}
	\item Il n'est pas possible de renvoyer plusieurs valeurs.
	\begin{minipage}{0.46\linewidth}
		\begin{minted}{python}
def f():
    a = 3
    b = 5
    return a, b # OK
		\end{minted}
	\end{minipage}
	\hfill
	\begin{minipage}{0.47\linewidth}
		\begin{minted}{cpp}
int f(){
   int a = 3, b = 5;
   return a, b; // ERREUR
}
		\end{minted}
	\end{minipage}
	\item On ne peut pas modifier les arguments (ils sont copiés).
	\begin{minipage}{0.46\linewidth}
			\begin{minted}{cpp}
void switch_1(double a,
              double b){
    // Échange a et b
    double c = b;
    b = a;
    a = c;
}
			\end{minted}
		\end{minipage}
		\hfill
		\begin{minipage}{0.51\linewidth}
			\begin{minted}{cpp}
int main(){
    double x = 5, y = 7;
    switch_1(x,y);
    cout << x << ", " << y;
    // Affiche "5, 7"
}
			\end{minted}
		\end{minipage}
	\end{itemize}
\end{frame}

\begin{frame}
	\frametitle{Le passage par référence}

	Le passage par référence est une solution aux problèmes précédents. Il autorise la modification de l'argument concerné.

	\begin{block}{Syntaxe}
	\texttt{type f(type1 {\huge\textbf{\&}} arg1, type2 {\huge\textbf{\&}} arg2, type3 arg3 ...)\{...\}}
	\end{block}
\end{frame}

\begin{frame}[fragile]
	\frametitle{Passage par référence : exemples}
	\begin{itemize}
	\item "Renvoyer" deux ou plus valeurs (modifier les arguments)
	\begin{minipage}{0.49\linewidth}
		\begin{minted}{cpp}
void f(int & a, int & b){
   a = 3;
   b = 5;
}
		\end{minted}
	\end{minipage}
	\hfill
	\begin{minipage}{0.47\linewidth}
		\begin{minted}{cpp}
int main(){
   int x, y;
   f(x, y);
   cout << x << " " << y;
   // Affiche "3, 5"
}
		\end{minted}
	\end{minipage}
	\item Modifier les arguments

	\begin{minipage}{0.49\linewidth}
			\begin{minted}{cpp}
void switch_2(double & a,
              double & b){
    // Échange a et b
    double c = b;
    b = a;
    a = c;
}
			\end{minted}
		\end{minipage}
		\hfill
		\begin{minipage}{0.47\linewidth}
			\begin{minted}{cpp}
int main(){
    double x = 5, y = 7;
    switch_2(x,y);
    cout << x << " " << y;
    // affiche 7 5
}
			\end{minted}
		\end{minipage}
	\end{itemize}
\end{frame}

\begin{frame}
\frametitle{Passage par référence : zoom sur la mémoire 1}
Fonctionnement de la fonction \texttt{switch\_1}
\centering
\only<1>{\includegraphics[width=\linewidth]{images/switch_1_00.pdf}}
\only<2>{\includegraphics[width=\linewidth]{images/switch_1_01.pdf}}
\only<3>{\includegraphics[width=\linewidth]{images/switch_1_02.pdf}}
\only<4>{\includegraphics[width=\linewidth]{images/switch_1_03.pdf}}
\only<5>{\includegraphics[width=\linewidth]{images/switch_1_04.pdf}}
\only<6>{\includegraphics[width=\linewidth]{images/switch_1_05.pdf}}
\only<7>{\includegraphics[width=\linewidth]{images/switch_1_06.pdf}}
\only<8>{\includegraphics[width=\linewidth]{images/switch_1_07.pdf}}
\only<9>{\includegraphics[width=\linewidth]{images/switch_1_08.pdf}}
\only<10>{\includegraphics[width=\linewidth]{images/switch_1_09.pdf}}
\end{frame}


\begin{frame}
\frametitle{Passage par référence : zoom sur la mémoire 2}
Fonctionnement de la fonction \texttt{switch\_2}
\centering
\only<1>{\includegraphics[width=\linewidth]{images/switch_1_00.pdf}}
\only<2>{\includegraphics[width=\linewidth]{images/switch_2_00.pdf}}
\only<3>{\includegraphics[width=\linewidth]{images/switch_2_01.pdf}}
\only<4>{\includegraphics[width=\linewidth]{images/switch_2_02.pdf}}
\only<5>{\includegraphics[width=\linewidth]{images/switch_2_03.pdf}}
\only<6>{\includegraphics[width=\linewidth]{images/switch_2_04.pdf}}
\only<7>{\includegraphics[width=\linewidth]{images/switch_2_05.pdf}}
\only<8>{\includegraphics[width=\linewidth]{images/switch_2_06.pdf}}
\only<9>{\includegraphics[width=\linewidth]{images/switch_2_07.pdf}}
\end{frame}

\begin{frame}[fragile]
	\frametitle{Surcharge}
	Il est possible de donner le même nom à deux fonctions (ou plus) à condition que les types ou le nombre des arguments de celles-ci soient différents.
	\vfill
	\begin{minipage}{0.47\linewidth}
	\begin{minted}{cpp}
int f(int v1, int v2){
    return v1 + v2;
}

double f(int v1, int v2){
    return 0.5;
}

// ERREUR !!!
// Même nom +
// arguments identiques
	\end{minted}
	\end{minipage}
	\hfill
	\begin{minipage}{0.47\linewidth}
	\begin{minted}{cpp}
int f(int v1, int v2){
    return v1 + v2;
}

double f(int v1, int v2, int v3){
    return 0.5;
}

// OK
// Même nom mais
// arguments différents
	\end{minted}
	\end{minipage}

\end{frame}

\begin{frame}[fragile]
	\frametitle{Portée, déclaration}

	\begin{block}{Appel de fonction}
		Comme pour les variables on ne peut utilisée une fonction que si elle a été déclarée préalablement.
	\end{block}
	\vfill

	\begin{minipage}{0.47\linewidth}

	\begin{minted}{cpp}
void f(){
    g(3);
    // ERREUR g est inconnue
}

void g(int i){
    ...
}
	\end{minted}
	\end{minipage}
	\hfill
	\begin{minipage}{0.47\linewidth}
	\begin{minted}{cpp}
void g(int i){
   ...
}

void f(){
    g(3); //OK
}
	\end{minted}
	\end{minipage}
\end{frame}

\begin{frame}[fragile]
	\frametitle{Portée, déclaration}

	\begin{minipage}{0.47\linewidth}
		\begin{minted}{cpp}
void f(){
    g(3);
    // ERREUR g est inconnue
}

void g(int i){
    f();
}
		\end{minted}
	\end{minipage}
	\hfill
	\begin{minipage}{0.47\linewidth}

		\begin{minted}{cpp}
void g(int i); // Déclaration

void f(){
    g(3); // OK

// Définition
void g(int i){
    ...
}
		\end{minted}
	\end{minipage}
\end{frame}

\begin{frame}[fragile]
	\frametitle{Variables locales et globales}

	\begin{block}{Variables globales}
		Les variables globales sont déclarées en dehors des fonctions.
		Elles sont accessibles à l'ensemble du programme.
	\end{block}

	\begin{minipage}{0.47\linewidth}
		\begin{minted}{cpp}
void f(){
    ...
    int i1 = 3;
    ...
}

void g(int i){
    cout << i1 << endl;
    /* ERREUR i1 inconnu
       i1 est une variable
       de f */
}
		\end{minted}
	\end{minipage}
	\hfill
	\begin{minipage}{0.47\linewidth}

		\begin{minted}{cpp}
int i1; // variable globale

void f(){
    ...
    i1 = 3;
    ...
}

void g(int i){
    cout << i1 << endl;
    /* OK i1 est une variable
       globale */
}
		\end{minted}
	\end{minipage}

\end{frame}


\begin{frame}[fragile]
	\frametitle{Variables locales et globales}

\begin{alert}{Attention}
	L'usage des variables globales est à limiter au maximum :
	\begin{itemize}
		\item La communication entre les fonctions est source de bugs.
		\item Elles rendent les fonctions difficilement réutilisables.
	\end{itemize}
\end{alert}

\begin{block}{Usage}
	Les variables globales sont des constantes générales.
\end{block}

\begin{minted}{cpp}
const int width = 800; // constante non modifiable
const int height = 600;

int main(){
    ...
    openWindow(width, height);
    ...
    height = 5; // ERREUR : height est une constante
}
\end{minted}

\end{frame}


\section{TP}

\begin{frame}{Organisation des TPs}
    \begin{itemize}
        \item En salle informatique de 9h45 à 11h15.
        \item Par groupe de 2
        \item Sur machine de l'école ou machine perso
        \item OS au choix
            \begin{itemize}
                \item Machine perso : Linux, Windows, OSX
                \item École : Windows, Linux
            \end{itemize}
        \item IDE au choix (QtCreator recommandé)
    \end{itemize}
\end{frame}

\begin{frame}{TP de la semaine}

	\begin{block}{Sujet}
        \begin{itemize}
            \item Utilisation du débogueur
            \item Utiliser Imagine++ pour jouer à Pong !
        \end{itemize}
	\end{block}

\end{frame}


\begin{frame}{Installation sur les portables}

	\begin{block}{Instructions}
		Elles sont accessibles via le site du cours :\\
		\textbf{\url{http://imagine.enpc.fr/~monasse/Info/}}
	\end{block}

	\begin{block}{Pour travailler chez vous}
		Installer un IDE et Imagine++ sur vos portables.
	\end{block}
	
	\begin{alertblock}{En cas de problème}
		Séance d'installation lundi 26 septembre à 16h45 \\
		Prendre rendez-vous avec Pascal Monasse
	\end{alertblock}

\end{frame}

\end{document}
